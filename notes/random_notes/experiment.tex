\documentclass{fancydoc}
\usepackage{amsmath}
\usepackage{amsthm}
\usepackage{hyperref}
\usepackage{todonotes}
%\hypersetup{
%	colorlinks=true,
%	linkcolor=blue,
%	filecolor=magenta,      
%	urlcolor=cyan,
%}

\newtheorem{mydef}{Definition}
\newtheorem{lemma}{Lemma}
\newtheorem{thm}{Theorem}
\newtheorem{note}{Note}
\newtheorem{ex}{Example}

\newcommand{\trans}{\mathrm{T}}
\newcommand{\diffd}{\mathrm{d}}
\newcommand{\realR}{\rm I\!R}
\newcommand{\tr}{\mathrm{tr}}

\DeclareMathOperator*{\argmax}{arg\,max}
\DeclareMathOperator*{\argmin}{arg\,min}

\begin{document}

\title{Random Notes}
\author{wxgopher}

\maketitle
\tableofcontents
\newpage

\dotfill


\section{Coordinate system transformation}
Notations:

Local coordinate system is denoted by $C_L$, and global~(world) coordinate system is denoted by $C_W$.

We define 
\begin{equation}
C_L = \begin{pmatrix}
e_1 & e_2 & e_3 & 0 \\
    &  0  &    & 1
\end{pmatrix} \in \realR^{4\times 4},
\end{equation}
which translates a homogeneous coordinate $x_L$ in $C_L$ to $x_W$ in $C_W$.

In $C_L$, we define two transformations~(rotation, translation, scaling) $T_1^L,T_2^L\in \realR^{4\times 4}$ of {\bf local coordinate system $C_L$ w.r.t. global coordinate system $C_W$}, and $T_1$ is applied before $T_2$. It's easy to derive in world space $C_W$, $T_1^{W\mapsto W} = C_L T_1^L C_L^{-1}$, and this is a mapping from global coordinate to global coordinate. Similarly, $T_1^{L\mapsto W} = T_1^{W\mapsto W} C_L = C_L T_1^L$.

Consider combining these transformations, we have (this time $C_L$ is actually $C_LT_1^L$, because we've moved our local coordinate system)
\begin{equation}
T_2\circ T_1 \equiv (T_2\circ T_1)^{L\mapsto W} = (C_L T_1^L) T_2 (C_L T_1^L)^{-1} C_L T_1^L = C_L T_1^L T_2^L,
\end{equation}
and to be clear, this mapping is applied on local coordinate system and produce global coordinate.


\section{Simple linear elasticity and implementation}
We use a variant of classical linear elasticity:

\begin{equation}
E(x) = \sum_t\frac{\mu}{2} \Vert D_s - D_m \Vert^2_\mathrm{F}
\end{equation}

where $D_s$ and $D_m$ are shape matrices, $D_m$ is a constant matrix, $D_s = G_t \cdot x$ where $G_t$ is a linear mapping (tensor) . Let $x = (x_1, x_2, \cdots, x_n)^\mathrm{T} \in \realR^{3n}$. We denote restpose with $x_m$ and deformed pose with $x_s$.

If we are to use a general optimizer like LBFGS (implemented in master branch), we need the gradient:

\begin{equation}
\nabla E(x_i) = \tr\left[(D_s - D_m)^{\mathrm{T}} \cdot \frac{\partial D_s}{\partial x_i}\right]
\end{equation}

The constraints is given by fixed boundaries on certain DoFs.

\subsection{Quadratic minimization}
Since $E$ is a quadratic form, we can convert this problem to a quadratic minimization. This is done as follows:

For every tet, we define $vec(D_{s}) = S_t \otimes I_3 \cdot x_s = Gx_s \in \realR^{9 \times 1}$, where $S_t$ is a selector matrix, defined by: $S_{t, ij} = 1$ iff $i = 1,2,3$ and $j$ is the index of that point, and $S_{t, ij} = -1$ iff $i=4$ and $j$ is the index of that point.

$G$ is a $9\times3n$ matrix, and $vec$ is a vectorization denoted as $vec((a_1, a_2, \cdots, a_n)) = (a_1, a_2, \cdots, a_n)^{\mathrm{T}}$.

Also note that $\tr(AB) = \tr(BA)$, $\tr(A+B) = \tr(A)+ \tr(B)$, and $\tr(A^{\mathrm{T}}) = \tr(A)$, then

\begin{subequations}
	\begin{align}
	 E(x) &= \sum \frac{\mu}{2} \Vert D_s - D_m \vert^2_F = \sum \frac{\mu}{2} \cdot \tr((D_s - D_m)\cdot(D_s - D_m)^\mathrm{T}) \newline \\
	 &= \sum \frac{\mu}{2}  \tr(D_s^{\mathrm{T}}D_s - 2D_s^{\mathrm{T}}D_m +D_mD_m^{\mathrm{T}}) 
	\end{align}
\end{subequations}
It's easy to derive:
\begin{subequations}
\begin{align}
\tr (D_s^\mathrm{T}D_s)& = x^\mathrm{T}G^\mathrm{T}Gx,\\
\tr(D_s^\mathrm{T} D_m) &= x^\mathrm{T}G^\mathrm{T} \cdot vec(D_m)= x^\mathrm{T}G^\mathrm{T} Gx_m.
\end{align}
\end{subequations}
Thus we have
\begin{equation}
E(x) =\frac{1}{2} x^\mathrm{T}Ax - x^\mathrm{T} b + c,
\end{equation}
where
\begin{subequations}
	\begin{align}
	A &= \mu\sum G^\mathrm{T}G \geq 0, \\
	 b& =\mu \sum G^\mathrm{T} Gx_m, \\
	  c &= \frac{\mu}{2} \sum \tr(D_mD_m^\mathrm{T}).
	\end{align}
\end{subequations}

The stationary point of $E$ can be obtained by solving $ Ax = b$.

\subsection{Constraints}
Two ways to impose $m$ constraints:

The constraint can be written as $Qx=q \Rightarrow x^{\mathrm{T}}Q^{\mathrm{T}} - q^{\mathrm{T}}=0$, where $Q\in R^{3m \times 3n}$, $q\in R^{3m \times 1}$, $m \leq n$.

\subsubsection{Soft constraints}
Let $w$ become a constraint parameter, the objective becomes
\begin{equation}
E(x) =\frac{1}{2}x^{\mathrm{T}}A x - x^{\mathrm{T}} b +c + w(x^{\mathrm{T}}Q^{\mathrm{T}}Qx - 2x^{\mathrm{T}}Q^{\mathrm{T}}q + q^{\mathrm{T}}q).
\end{equation}
The final linear system to minimize $E$ becomes solving
\begin{equation}
 ( A + 2wQ^{\mathrm{T}}Q)x = b + 2 wQ^{\mathrm{T}}q.
\end{equation}

\subsubsection{Lagrangian multipliers}
The Lagrangian of $E$ is
\begin{equation}
 L(x, \lambda) = \frac{1}{2} x^\mathrm{T}Ax - b^\mathrm{T}x + c + \lambda^{\mathrm{T}}(Qx - q) \newline = \frac{1}{2} x^\mathrm{T}Ax + ( \lambda^\mathrm{T} Q -b^\mathrm{T})x + c-\lambda^\mathrm{T}q.
\end{equation}

Let
\begin{equation}
\frac{\partial L}{\partial x} = Ax + Q^{\mathrm{T}}\lambda - b = 0.
\end{equation}
Hence the dual form of the problem:
\begin{equation}
\begin{pmatrix} A&Q^{\mathrm{T}} \\ Q & 0 \end{pmatrix} \cdot \begin{pmatrix} x\\ \lambda \end{pmatrix} = 
\begin{pmatrix} b \\ q \end{pmatrix} 
\end{equation}
Tips for implementation:
Creating sparse matrices in a loop is costly. An efficient evaluation of $A$:
\begin{equation}
 A = \mu \sum_i (G^{\mathrm{T}}_iG_i) \newline = \sum_i (S_i \otimes I_3)^\mathrm{T}(S_i \otimes I_3) \newline = (\sum_i S^\mathrm{T}_i S_i) \otimes I_3.
\end{equation}

Precomputing $(\sum_i S^\mathrm{T}_i S_i)$ can be done via coo\_matrix (in scipy) on $(i,j,k)$ tuples.

\section{Position-based Dynamics~(PBD) related}

\subsection{Angle-based triangle bending constraint}

Just came across an implementation in $\mathrm{Magica}$~(a Unity plugin).

\bibliographystyle{plain}
\bibliography{bib.bib} 

\end{document}